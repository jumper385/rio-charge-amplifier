% IEC 61000-4-2 ESD IMMUNITY TEST REPORT
% =============================================================================
\documentclass[a4paper,12pt]{article}

% ---------- Packages ----------
\usepackage[margin=25mm]{geometry}
\usepackage{graphicx}
\usepackage{booktabs}
\usepackage{longtable}
\usepackage{array}
\usepackage{hyperref}
\usepackage{siunitx}
\usepackage{fancyhdr}

% ---------- Header / Footer ----------
\pagestyle{fancy}
\fancyhf{}                                         % clear defaults
\lhead{IEC 61000-4-2 Test Report}
\rhead{RioTinto Charge Amplifier}
\cfoot{\thepage}

% ---------- Metadata ----------
\title{\bfseries ESD Immunity Test Report\\[2mm]
        (IEC 61000-4-2:2014)}
\author{Quantum Digital Labs - Bayswater, Western Australia}

% ---------- Convenient column types ----------
\newcolumntype{L}{>{\raggedright\arraybackslash}p{0.35\linewidth}}
\newcolumntype{R}{>{\raggedright\arraybackslash}p{0.60\linewidth}}

% ----------------------------------------------------
\begin{document}
\maketitle
\thispagestyle{empty}
\newpage
\tableofcontents
\newpage

% =============================================================================
\section{Scope and Reference Documents}
In accordance with EN 61326-1 Clause 7, this test report aims to verify that the Equipment Under Test (EUT) complies with the IEC 61000-4-2 (ESD Immunity) requirements. Device performance is evaluated against Criterion B, which allows for a temporary loss of function provided the system recovers within 60 seconds. Compliance is demonstrated by comparing pre-test and post-test measurements obtained with a reference charge simulator, and the similarity of the two data sets is quantified using the Wasserstein distance metric.

% =============================================================================
\section{Equipment Under Test (EUT)}
\begin{table}[h]
\centering
\begin{tabular}{L R}
      Product Name & Charge Amplifier Board \\
      Manufacturer & RioTinto \\
      Hardware Revision & V1.0 \\
      Power Input & 24 V DC $\leq$ 30 mA \\
\end{tabular}
\end{table}

% =============================================================================
\section{Test Equipment}
\begin{table}[h]
\centering
\begin{tabular}{|l|l|l|l|l|}
      ID & Equipment & Function & Calibration Date & Calibration Due \\
      \hline
      E01 & ESD Gun & ESD Generation & YYYY-MM-DD & YYYY-MM-DD\\
      E02 & HCP and VCP Plane & Coupling Planes & - & - \\
      E03 & Environmental Sensor & Environmental Monitoring & YYYY-MM-DD & YYYY-MM-DD \\
\end{tabular}
\end{table}

% =============================================================================
\section{Climatic and Environmental Conditions}

\begin{center} % keeps the table centred like your screenshot
\begin{tabular}{|l|c|c|}
  \hline
  \textbf{Parameter} &
  \textbf{Standard Range (IEC 61000-4-2)} &
  \textbf{Measured}\\
  \hline
  Ambient Temperature & 15$^\circ$C - 35$^\circ$C & XX$^\circ$C\\
  Relative Humidity   & 30 \% - 60 \%              & XX \%\\
  Atmospheric Pressure& 86 kPa - 106 kPa           & XXX kPa\\
  \hline
\end{tabular}
\end{center}

% =============================================================================
\section{Special Test Conditions and Setup}
\begin{itemize}
      \item{Horizontal Coupling Plane: 0.5m x 0.5m place placed 0.1m from closest EUT edge}
      \item{Vertical Coupling Plane: 0.5m x 0.5m placed 0.1m from closest EUT edge}
\end{itemize}

\subsection{Device Operating Mode}
\begin{itemize}
      \item{Device to be set to test mode to enable the on-board MCU to capture samples from the device}
\end{itemize}

% =============================================================================
\section{Test Procedure (IEC 61000-4-2)}
% nubmered list
% pre test daq
%       turn on device and set it into test mode
%       connect to device via the auxiliary gpio connector to get uart comunications 
%       initialize device into data logging mode to begin data stream in 10 seconds
%       disconnect from the device and leave running for 1.5 minutes. 
%       connect back and extract the last 1 minute of data (0.4 min to 1.4 min)

% apply the test
%       continue measuring data from the eut and apply esd pulses as required. 
%       after each application of ESD pulses, wait 2 minutes
%       connect back to the device and retreive the last 1 minute of data

% confirm wasserstein distance
%       measure the wassersteing distance between the pre and post test samples.
%       pass if wasserstein distance is 0.2

\begin{enumerate}
      \item{Pre-test DAQ Setup:}
            \begin{itemize}
                  \item{Turn on device and set it into test mode.}
                  \item{Connect to device via the auxiliary GPIO connector to enable UART communications.}
                  \item{Initialize device into data logging mode to begin data stream in 10 seconds.}
                  \item{Disconnect from the device and leave running for 1.5 minutes.}
                  \item{Reconnect and extract the last 1 minute of data (0.4 min to 1.4 min).}
            \end{itemize}

      \item{ESD Application:}
            \begin{itemize}
                  \item{Continue measuring data from the EUT and apply ESD pulses as required.}
                  \item{After each application of ESD pulses, wait 2 minutes.}
                  \item{Reconnect to the device and retrieve the last 1 minute of data.}
            \end{itemize}

      \item{Wasserstein Distance Confirmation:}
            \begin{itemize}
                  \item{Measure the Wasserstein distance between the pre-test and post-test samples.}
            \end{itemize}
\end{enumerate}

Apply tests to the following locations:
\begin{center}
      \begin{tabular}{|l|L|}
            \hline
            Location & Description \\
            \hline
            Top and Bottom & 0.5m x 0.5m plane placed 0.1m from closest EUT edge \\
            Vertical Coupling Plane & 0.5m x 0.5m plane placed 0.1m from closest EUT edge \\
            Direct Contact & Direct contact with ESD gun at various points on the EUT \\
            \hline
      \end{tabular}
\end{center}

| location | description |

% =============================================================================
\section{Performance Criterion and Decision Rules}
The performance criterion for this test is Criterion B, which allows for temporary loss of function provided the system recovers within 60 seconds. Compliance is determined by comparing pre-test and post-test measurements obtained with a reference charge simulator. The similarity of the two data sets is quantified using the Wasserstein distance metric, with a threshold of 0.2 indicating compliance.

% =============================================================================
\section{Results}
% | test id | application point | wasserstein distance | pass/fail |
\begin{table}[h]
      \centering
      \begin{tabular}{|l|l|l|l|}
            \hline
            Test ID & Application Point & Wasserstein Distance & Result \\
            \hline
            ESD-001 & Horizontal Coupling Plane & 0.15 & Pass \\
            ESD-002 & Vertical Coupling Plane & 0.18 & Pass \\
            ESD-003 & Direct Contact & 0.20 & Pass \\ 
            \hline
      \end{tabular}
      \caption{Test Results Summary}
      \label{tab:test_results}
\end{table}


% =============================================================================
\section{Specific Conditions of Use}

% =============================================================================
\section{Photographs}

% =============================================================================
\section{Conclusions}
\newpage
\appendix
\section{Baseline and Post-Test Functional Data}\label{app:baseline}

\section{Oscilloscope Waveforms}\label{app:waveforms}

\section{Calibration Certificates}\label{app:cal}

\end{document}

